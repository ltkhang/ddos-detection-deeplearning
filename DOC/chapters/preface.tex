Trong thời đại phát triển nhanh chóng của Internet hiện nay, các phương thức bảo mật và an toàn thông tin đang ngày càng được chú trọng. Trong đó, tấn công từ chối dịch vụ (DoS) và tấn công từ chối dịch vụ phân tán (DDoS) đang trở thành mối đe dọa lên toàn thể các máy chủ trên thế giới. Từ đó, các phương thức và dịch vụ phòng chống, giảm thiểu thiệt hại cũng được ra mắt với rất nhiều các chức năng khác nhau.

Một số vấn đề nổi trội khi thực hiện phòng thủ trước tấn công DoS/DDoS đó chính là chi phí cài đặt và vận hành hệ thống khá tốn kém. Bên cạnh đó, các phương pháp tấn công ngày càng đa dạng với quy mô ngày một lớn, khiến cho các phương pháp phòng chống nhanh chóng trở nên lỗi thời và bị vô hiệu hóa. Trong khóa  luận này, tôi hi vọng có thể giải quyết được các vấn đề trên, cung cấp một giải pháp phòng thủ trước tấn công DoS/DDoS trong mạng Software defined network (SDN) hợp lý, gọn nhẹ và dễ dàng triển khai.

Tôi đã tiến hành tìm hiểu các nghiên cứu học thuật cũng như một số các sản phẩm liên quan đến vấn đề này trên thị trường. Từ đó, xem xét khả năng ứng dụng học sâu và học máy vào giải pháp phát hiện tấn công DoS/DDoS. Sau đó tôi thiết kế và xây dựng một proof-of-concept cho phần mềm nhận diện tấn công DoS/DDoS và một proof-of-concept cho hệ thống triển khai trong mạng SDN.

Văn bản này trình bày tổng kết lại quá trình tìm hiểu và xây dựng Hệ thống phát hiện và phòng thủ trước các cuộc tấn công từ chối dịch vụ và từ chối dịch vụ phân tán trong môi trường mạng SDN. Toàn bộ văn bản được chia thành các chương với nội dung như sau:

\textbf{Chương \ref{chap:introduction}. Giới thiệu}

Trình bày chi tiết nguyên nhân và động lực của tôi khi thực hiện khóa luận này, cũng như mục tiêu đặt ra và hướng tiếp cận của tôi đối với vấn đề.

\textbf{Chương \ref{chap:background}. Cơ sở lý thuyết}

Trình bày những khái niệm cơ bản, giúp người đọc có cái nhìn tổng quan về cơ sở lý thuyết của khóa luận cũng như làm nền tảng cho những chương tiếp theo.

\textbf{Chương \ref{chap:review}. Các công trình liên quan}

Trình bày những tìm hiểu của tôi về các nghiên cứu phát hiện và phòng thủ trước tấn công DoS/DDoS đã có. Tìm ưu nhược điểm trong các nghiên cứu này.

\textbf{Chương \ref{chap:method}. Mô hình giải pháp của đề tài}

Trình bày và giới thiệu giải pháp của tôi đề xuất cho một hệ thống phát hiện và phòng thủ trước tấn công DoS/DDoS.

\textbf{Chương \ref{chap:network_model}. Mô hình mạng thử nghiệm}

Áp dụng giải pháp đề xuất ở chương 4 vào mạng SDN trong môi trường mạng thử nghiệm.

\textbf{Chương \ref{chap:testing}. Thử nghiệm và phân tích kết quả}

Kiểm lỗi và độ chính xác của hệ thống khi áp dụng vào mạng SDN

\textbf{Chương \ref{chap:conclusion}. Tổng kết}

Tổng kết lại quá trình thực hiện khóa luận, những điểm đạt và chưa đạt, đồng thời đề xuất hướng phát triển trong tương lai. 